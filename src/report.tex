\documentclass[12pt]{article}
% We can write notes using the percent symbol!
% The first line above is to announce we are beginning a document, an article in this case, and we want the default font size to be 12pt
\usepackage[utf8]{inputenc}
% This is a package to accept utf8 input.  I normally do not use it in my documents, but it was here by default in Overleaf.
\usepackage{amsmath}
\usepackage{amssymb}
\usepackage{amsthm}
% These three packages are from the American Mathematical Society and includes all of the important symbols and operations 
\usepackage{fullpage}
% By default, an article has some vary large margins to fit the smaller page format.  This allows us to use more standard margins.

\setlength{\parskip}{1em}
% This gives us a full line break when we write a new paragraph


\begin{document}
% Once we have all of our packages and setting announced, we need to begin our document.  You will notice that at the end of the writing there is an end document statements.  Many options use this begin and end syntax.

\begin{center}
    Laboratory Report \\
    Rohit Raj
\end{center}

\begin{center}
    \Large The EAP-TLSv1.3 connection behavior
\end{center}

The latest Transport Layer Security (TLS) 1.3 was released in August 2013. The arrival of this new version, has brought some exciting new features. 

\section{Burning Questions!}

\begin{enumerate}
	\item{Will the PSK be common for all the devices?}
	\item{Will the key renegotiation be a part of TLS1.3 architecture, or are we supposed to do something else?}
	\item{Starting point of implementation. \emph{wpa\_supplicant} or \emph{hostap}? At what point will we deviate from standard EAP-TLS (purely in implementational sense)?}
	
\end{enumerate}

\end{document}